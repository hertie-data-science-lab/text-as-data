\documentclass{hertieteaching}
\usepackage{pifont}

\title{Text as Data as Measurement}

\begin{document}

\maketitle

% model structure choices
% preprocessing
% strategic considerations

\begin{frame}{Last week}

Last week we talked rather abstractly about models that connected the `message' $\theta$ and the words $W$ (or whatever features we decided to treat as exchangeable)

Let's be a bit more specific

\end{frame}


\begin{frame}{Decisions, decisions}

Are we modeling 
\begin{itemize}
  \item the generation process
  \item the understanding process
  \item or maybe both\ldots
\end{itemize}

\pause

Por qué no los dos?
\begin{align*}
P(\theta) && \textit{Prior expectations}\\
P(\{W\} \mid \theta) && \textit{Generation} \\
P(\theta \mid \{W\}) & = \frac{P(\{W\} \mid \theta)P(\theta)}{\int P(\{W\} \mid \theta) P(\theta) d\theta} & \textit{Understanding}
\end{align*}

\end{frame}
\begin{frame}{Decisions, decisions}
Examples:

Document classification: $\theta$ is the probability that this document is about social policy
\begin{itemize}
  \item Naive Bayes Classification, learn all the things
  \item (Regularized) Logistic Regression, go straight for $P(\theta \mid \{W\})$ 
\end{itemize}

Thematic analysis: $\theta$ is the proportion of social policy mentions in the document
\begin{itemize}
  \item Topic Models, learn all the things
  \item Content Analysis Dictionaries, assert $P(\{W\} \mid \theta)$ and go straight for $P(\theta \mid \{W\})$
\end{itemize}

We'll take a closer look at thematic analysis next week, so let's look at classification

\end{frame}
\begin{frame}{Document classification}

Naive Bayes:

Let $Z$ be one of \textit{two} possible document topics
\begin{align*}
P(\{W\} \mid Z) & = \prod^V P(W_v \mid Z) & \textit{The naive part}\\
P(Z=1) & = \theta
\end{align*}
Now we see a 
  
\end{frame}

\begin{frame}[t,fragile]\frametitle{Naive Bayes}

Estimating the probability that a word profile $\{W\}_j$ occurs given that the document is liberal $P(\{W\}_j\mid Z=\text{`Lib'})$ is more challenging, because any one word profile is likely to occur only once.

Assumption: words are assumed to be generated \textit{independently} given the category Z
\begin{align*}
P(\{W\}_j \mid Z=\text{`Lib'}) &~=~ {\prod}_i P(W_i \mid Z=\text{`Lib'})\\
P( \text{`Affirmative Action'} \mid Z=\text{`Lib'}) &~=~ P( \text{`Affirmative'} \mid Z=\text{`Lib'}) \cdot\\
& ~  P( \text{`Action'} \mid Z=\text{`Lib'})
\end{align*}


\end{frame}



\begin{frame}[t,fragile]\frametitle{Naive Bayes}

With this assumption, we can estimate the probability of observing a word $i$  given that the document is liberal: proportion of word $i$ in liberal training set.

The classifier then chooses the class Z (Liberal or Conservative) with the highest aggregate probability.

Note that every new word adds a bit of information that re-adjusts the conditional probabilities.

\newpage


\end{frame}
\begin{frame}[t,fragile]\frametitle{Naive Bayes}



Note that with two classes (here: liberal and conservative)  this has a rather neat interpretation:
\begin{align*}
\frac{P(Z=\text{\text{`Lib'}} \mid \{W\}_j)}
{P(Z=\text{`Con'} \mid \{W\}_j)} = \\
~~~~~~~~~\prod_i \frac{P(W_i \mid Z=\textsl{\text{`Lib'}})}
{P(W_i \mid Z=\textsl{`Con'})}\times \frac{P(Z=\textsl{\text{`Lib'}})}{~P(Z=\textsl{`Con'})}
\end{align*}
Logging this probability ratio, every new word \textsl{adds} a bit of information that pushes the ratio above or below 0

%%%%%%%%%%% check the below works in the real document

%[will@Apparent lib]$ ykcats -dictionary ../../disc.vbpro ../lib/*
%,Dictionary,Dictionary>discrimination,WordCount
%6019_ll01-utf8.txt,26,26,20002
%6020_ll01-utf8.txt,13,13,18722
%[will@Apparent lib]$ ykcats -dictionary ../../disc.vbpro ../con/*
%,Dictionary,Dictionary>discrimination,WordCount
%6019_lc01-utf8.txt,70,70,17368
%6020_lc01-utf8.txt,48,48,17698

% lib = (26+13)/(20002+18722) = 0.001007127
% con = (70+48)/(17368+17698) = 0.003365083
% lib / con = .299

%%%%%%%%%%%

\end{frame}
\begin{frame}[t,fragile]\frametitle{Naive Bayes}

Example: Naive Bayes with only word class `discriminat*'.
{\small
\begin{align*}
P(W=\text{`discriminat*'} \mid Z=\text{`Lib'}) & = (26+13)/(20002+18722) \approx 0.001\\
P(W=\text{`discriminat*'} \mid Z=\text{`Con'}) & = (70+48)/(17368+17698) \approx 0.003
\end{align*}
}
Assume that liberal and conservative supporting briefs are equally likely (true in the training set)
\begin{align*}
\frac{P(Z=\text{\text{`Lib'}})}{P(Z=\text{`Con'})} & = 1
\end{align*}

Last step:  calculate posterior classification  probabilities for a new document (based on occurrence of this word).

\end{frame}
\begin{frame}[t,fragile]\frametitle{Naive Bayes}

%(File 6019\_al18-utf8.txt)
Amicus brief from `King County Bar Association' containing 3667 words and 4 matches to disciminat*.

{\scriptsize
\begin{verbatim}
that "the state shall not [discriminate] against, or grant preferential treatment
the lingering effects of racial [discrimination] against minority groups in this
remedy the effects of societal [discrimination]. Another four Justices (Stevens
that "the state shall not [discriminate] against, or grant preferential treatment
\end{verbatim}
}
\end{frame}


\begin{frame}[t,fragile]\frametitle{Naive Bayes}
A priori, the probabilities are\ldots

Probability that we observe the word  discriminat* 4 out of 3667 times if the document is liberal:
{\small
\begin{verbatim}
> dbinom(4, size=3667, prob=0.001007127)
[1] 0.1930602
\end{verbatim}
}
Probability that we observe the word  discriminat* 4 out of 3667 times if the document is conservative:
{\small
\begin{verbatim}
> dbinom(4, size=3667, prob=0.003365083)
[1] 0.004188261
\end{verbatim}
}
Logged probability ratio = 3.83
\newpage
\end{frame}


\begin{frame}[t,fragile]\frametitle{Naive Bayes}

Conclusion: Seeing 4 instances of discriminat* gives the posterior classification probabilities
\begin{itemize}
\item $\theta_\text{liberal} = \frac{0.193}{0.193+0.004}$ = 0.979
\item $\theta_\text{conservative}$ =  1-0.979=0.021
\end{itemize}

This is \textit{quite} confident
\begin{itemize}
\item \ldots but other words will be less loaded or push the other way
\end{itemize}
\end{frame}



\end{document}
