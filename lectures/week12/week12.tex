\documentclass{hertieteaching}
\usepackage{cancel}
\usepackage{hyperref}

\title{Text in research context}

\begin{document}

\maketitle

\begin{frame}{Where to put it?}

Simple lexical analysis
\begin{itemize}
  \item $C_\mathit{ij}$ as an \textit{indicator} or an effect of something non-textual
\end{itemize}
\pause

We've focused more on text analysis as a \textit{measurement problem}:
\begin{itemize}
  \item (Documents + assumptions) $\Longrightarrow$ $\hat{\theta}$
\end{itemize}

\begin{columns}[T,onlytextwidth]
\column{0.5\textwidth}

Where does this fit in the larger research picture?
\begin{itemize}
  \item $\theta$ as an independent variable
  \item $\theta$ as an dependent variable
  \item $\theta$ as a confounder
\end{itemize}

\pause
Remember, there's still measurement error, even if there isn't bias

\column{0.05\textwidth}
\column{0.45\textwidth}

\begin{tikzpicture}
\node(W1) at  (0.5, 0)[var];
\node(W2) at  (1, 0)[var];
\node(W3) at  (1.5, 0)[var];
\node(theta) at  (1, 1)[lat,label=left:$\theta_X$]{};
\draw  (theta) -- (W1);
\draw  (theta) -- (W2);
\draw  (theta) -- (W3);
\pause
\node(Y1) at  (3, 0)[var];
\node(Y2) at  (3.5, 0)[var];
\node(Y3) at  (4, 0)[var];
\node(thetaY) at  (3.5, 1)[lat,label=right:$\theta_Y$]{};
\draw  (thetaY) -- (Y1);
\draw  (thetaY) -- (Y2);
\draw  (thetaY) -- (Y3);
\draw  (theta) -- (thetaY);
\pause
\node(Z) at  (2.25, 2)[var,label=left:$Z$];
\draw  (Z) -- (thetaY);
\draw  (Z) -- (theta);
\end{tikzpicture}

\end{columns}


\end{frame}


\begin{frame}{Strategy}

We can also think of a text analysis
\begin{itemize}
  \item[1.] $\theta$ as a large scale terrain map / sample stratifier
  \item[2.] $\theta$ as a generalization check
\end{itemize}
  
Examples of 1 and 2:
\begin{itemize}
  \item Classify / scale / topic model 10,000 news stories and use $\hat{\theta}$ to see which ones to read more closely 
  \item Work up a small dictionary on 30 documents and
  apply to the 10,000 news stories 
\end{itemize}
  
Examples of iteration:
\begin{itemize}
  \item Work up a small dictionary on 30 stories
  \item Apply to the 10,000 stories to see macro trends
  \item Sample interesting, extreme, of randomly based on $\theta$ to check the model
\end{itemize}


  
\end{frame}

\begin{frame}{Tactics}

Sampling?
\begin{itemize}
  \item Who  or what is the population?
  \item Down-sampled data means you iterate models faster (and risk missing something)
  \item Thoughtful stratification will help you draw more robust conclusions
\end{itemize}

Model checking?
\begin{itemize}
  \item How you would \textit{check} the model, e.g. for stability
  of the $\theta \longrightarrow W$ mapping or the $\beta$s
\end{itemize}

Tools
\begin{itemize}
  \item Much text analysis is inherently mechanical / automated: 
  be very instrumental about packages and tools
  \item Don't be afraid to ask for help: e.g. me, the Data Science Lab's research consulting service
\end{itemize}

\end{frame}

\begin{frame}{Hints and Tips}

Always try the Kartoffelpuffer (ideally with apple sauce)
\begin{itemize}
  \item Very unhealthy, but quite yummy
\end{itemize}

\bigskip
\pause
For those of you currently outside Germany


Never eat Kartoffelpuffer (particularly with apple sauce)
\begin{itemize}
  \item Terrible. Especially with Glühwein. 
  \item You're totally not missing anything.
\end{itemize}

\end{frame}

\begin{frame}{~~}

\bigskip
\centerline{\includegraphics[scale=0.4]{pictures/thatsallfolks}}

\end{frame}


%\begin{frame}[allowframebreaks]
%\frametitle{References}
%\printbibliography	
%\end{frame}

\end{document}